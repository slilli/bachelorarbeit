% !TeX encoding = UTF-8
\section{Einführung}

Das Einführungskapitel beinhaltet ein paar praktische Hinweise zum Schreiben
der Abschlussarbeit, sowie eine Kurzdokumentation der bereitgestellten
\LaTeX-Klasse.
Die restlichen Kapitel dienen lediglich zu Demonstrationszwecken.

\subsection{Zur Abschlussarbeit als solche}

Neben der jeweiligen Studien- und Prüfungsordnung, die die förmlichen
Eigenschaften der Durchführung einer Abschlussarbeit regelt, sind folgende
Quellen hilfreich:
\begin{itemize}
    \item Studien- und Prüfungsordnungen der Informatikstudiengänge
    (\url{http://www.imp.fu-berlin.de/fbv/pruefungsbuero/Studien--und-Pruefungsordnungen/index.html})

    \item ThesisRules (\url{http://www.inf.fu-berlin.de/w/SE/ThesisRules}):

    Beschreibung des praktischen Ablaufs einer Abschlussarbeit in der AG
    Software Engineering von A bis Z.

    \item "`Technisches Schreiben"'
    (\url{http://www.mi.fu-berlin.de/wiki/pub/SE/SeminarRegeln/Technisches_Schreiben.pdf}):

    Ein von Lutz Prechelt verfasstes Dokument mit vielen praktischen
    Hinweisen zum Schreibteil (nicht nur) einer Abschlussarbeit.
\end{itemize}

\subsection{Zu dieser \LaTeX{}-Vorlage}

\subsubsection{Optionen der Dokumentenklasse}

Die Dokumentenklasse \texttt{agse-thesis} unterstützt verschiedene
Schriftarten:
\begin{lstlisting}[language={[LaTeX]TeX}]
% Standard LaTeX Schriftart
\documentclass[plain]{agse-thesis}

% Serifenschrift Palladino
\documentclass[serif]{agse-thesis}

% Serifenlose Schrift Paratype Sans
\documentclass[sans-serif]{agse-thesis}
\end{lstlisting}

Für kürzere Arbeiten, die mit Abschnitten (\texttt{\textbackslash{}section})
als oberste Gliederungsebene auskommen, reicht die Standard-Option
\texttt{article}.
Die Buch-Option \texttt{book} bietet darüber hinaus noch Kapitel
(\texttt{\textbackslash{}chapter}) an.
\begin{lstlisting}[language={[LaTeX]TeX}]
% Standard fuer kuerzere Arbeiten
\documentclass[article]{agse-thesis}

% Buch-Variante fuer umfangreiche Arbeiten mit vielen
% Gliederungselementen
\documentclass[book]{agse-thesis}
\end{lstlisting}

Ob zwischen den Absätzen im Text Abstände angezeigt werden sollen, oder ob
stattdessen die erste Zeile eines Absatzes eingerückt werden soll, kann mit
\texttt{parskip} bzw. \texttt{noparskip} eingestellt werden.
\begin{lstlisting}[language={[LaTeX]TeX}]
% Absaetze deutlich trennen
\documentclass[parskip]{agse-thesis}

% Absaetze nah bei einander, erste Zeile eingerueckt
\documentclass[noparskip]{agse-thesis}
\end{lstlisting}

Die Dokumentensprache kann Deutsch oder Englisch sein (bitte mit Betreuer/in
absprechen). Die Angabe der Sprache ist wichtig, damit \LaTeX\ u.a. die
Silbentrennung und die Benennung von Überschriften und Bezeichnungen
(für Abbildungen und Tabellen) richtig behandeln kann.
\begin{lstlisting}[language={[LaTeX]TeX}]
% Deutsch
\documentclass[de]{agse-thesis}

% Englisch
\documentclass[en]{agse-thesis}
\end{lstlisting}

Die Werte der vier o.g. Optionen können beliebig kombiniert werden:
\begin{lstlisting}[language={[LaTeX]TeX}]
% Einstellung des Beispieldokuments
\documentclass[serif,article,noparskip,de]{agse-thesis}
\end{lstlisting}

\subsubsection{Befehl \texttt{\textbackslash{}thesisTitle}}

Der Titel der Arbeit wird sowohl auf der Titelseite (siehe
\ref{sec:cmd-coverpage}) als auch für die PDF-Metainformationen benötigt.
Gesetzt wird der Titel durch das Definieren von
\texttt{\textbackslash{}thesisTitle}.

Für die Titelseite können manuell mit \texttt{\textbackslash{}par}
Zeilenumbrüche eingefügt werden um das Textbild zu verbessern (nicht hingegen
mit \texttt{\textbackslash\textbackslash}).
Sollte von dieser Möglichkeit Gebrauch gemacht werden, muss der Titel für die
PDF-Metainformationen manuell gesetzt werden
(\texttt{\textbackslash{}hypersetup\{pdftitle=\{...\}\}}).


\subsubsection{Befehl \texttt{\textbackslash{}coverpage}}
\label{sec:cmd-coverpage}

Die Titelseite der Abschlussarbeit wird mit dem
\texttt{\textbackslash{}coverpage}-Befehl erzeugt.
Dessen Ausgabe wird über eine Reihe von Schlüssel-Wert-Paaren konfiguriert
(siehe \autoref{tab:coverpage-config}).
\begin{table}[ht]
\begin{center}
    \begin{tabular}{|l|L{5.5cm}|L{4cm}|}
        \hline
        \textbf{Schlüssel} & \textbf{Funktion} & \textbf{Default-Wert} \\
        \hline
        \texttt{student/id} & Matrikel-Nummer & -- \\
        \texttt{student/mail} & E-Mail-Adresse & -- \\
        \texttt{thesis/type} & Art der Abschlussarbeit & "`Bachelorarbeit"' \\
        \texttt{thesis/group} & Arbeitsgruppe in der die Arbeit geschrieben
        wurde & "`Arbeitsgruppe Software Engineering"' \\
        \texttt{thesis/advisor} & \emph{optional:} Betreuer der Abschlussarbeit
        & -- \\
        \texttt{thesis/examiner} & Erstgutachter der Arbeit & -- \\
        \texttt{thesis/examiner/2} & \emph{optional:} Zweitgutachter der Arbeit
        & -- \\
        \texttt{thesis/date} & \emph{optional:} Datum der Abgabe & aktuelles
        Datum\\
        \texttt{title/size} & \emph{optional:} \LaTeX-Schriftgröße für den
        Titel (\zb \texttt{\textbackslash{}LARGE}) & wird automatisch gesetzt \\
        \texttt{abstract/separate} & \emph{optional:} Schlüssel ohne Wert;
        falls gesetzt, wird der Abstract auf eine eigene Seite gesetzt und die
        Titelseite ist "`luftiger"' & -- \\
        \hline
    \end{tabular}
    \caption{Schlüssel-Wert-Konfiguration des
    \texttt{\textbackslash{}coverpage}-Kommandos.}
    \label{tab:coverpage-config}
\end{center}
\end{table}
Das einzige Argument des Kommandos ist der Abstract der Arbeit.
Ein minimaler Aufruf könnte so aussehen:
\begin{lstlisting}[language={[LaTeX]TeX}, morekeywords={coverpage}]
\coverpage[
    student/id=1234567,
    student/mail=email@inf.fu-berlin.de,
    thesis/type=Masterarbeit,
    thesis/examiner={Prof. Dr. Mia Maus}
]
{
    Prokrastination ist ein gut verstandenes Verhalten,
    das auch vor Abschlussarbeitern mit Informatik-Hintergrund
    nicht halt macht.
    % ...
}
\end{lstlisting}


\subsubsection{Verbesserungen der \LaTeX-Vorlage}

Diese \LaTeX-Vorlage soll den Einstieg in das Setzen der Abschlussarbeit
erleichtern.
Die Vorlage selbst wird in einem öffentlichen Git-Repository in der
GitLab-Instanz des Fachbereiches verwaltet, welches gerne als Grundlage für die
eigene Ausarbeitung geklont werden darf:
\begin{lstlisting}[language=bash]
git clone https://git.imp.fu-berlin.de/agse/thesis-template
\end{lstlisting}
Änderungsvorschläge in Form von Merge-Requests sind jederzeit willkommen.
